
        \documentclass[spanish, 11pt]{exam}

        %These tell TeX which packages to use.
        \usepackage{array,epsfig}
        \usepackage{amsmath, textcomp}
        \usepackage{amsfonts}
        \usepackage{amssymb}
        \usepackage{amsxtra}
        \usepackage{amsthm}
        \usepackage{mathrsfs}
        \usepackage{color}
        \usepackage{multicol, xparse}
        \usepackage{verbatim}


        \usepackage[utf8]{inputenc}
        \usepackage[spanish]{babel}
        \usepackage{eurosym}

        \usepackage{graphicx}
        \graphicspath{{../img/}}
        \usepackage{pgf}



        \printanswers
        \nopointsinmargin
        \pointformat{}

        %Pagination stuff.
        %\setlength{\topmargin}{-.3 in}
        %\setlength{\oddsidemargin}{0in}
        %\setlength{\evensidemargin}{0in}
        %\setlength{\textheight}{9.in}
        %\setlength{\textwidth}{6.5in}
        %\pagestyle{empty}

        \let\multicolmulticols\multicols
        \let\endmulticolmulticols\endmulticols
        \RenewDocumentEnvironment{multicols}{mO{}}
         {%
          \ifnum#1=1
            #2%
          \else % More than 1 column
            \multicolmulticols{#1}[#2]
          \fi
         }
         {%
          \ifnum#1=1
          \else % More than 1 column
            \endmulticolmulticols
          \fi
         }
        \renewcommand{\solutiontitle}{\noindent\textbf{Sol:}\enspace}

        \newcommand{\samedir}{\mathbin{\!/\mkern-5mu/\!}}

        \newcommand{\class}{2º Bachillerato}
        \newcommand{\examdate}{\today}

        \newcommand{\tipo}{A}


        \newcommand{\timelimit}{50 minutos}



        \pagestyle{head}
        \firstpageheader{\includegraphics[width=0.2\columnwidth]{header_left}}{\textbf{Departamento de Matemáticas\linebreak \class}\linebreak \examnum}{\includegraphics[width=0.1\columnwidth]{header_right}}
        \runningheader{\class}{\examnum}{Página \thepage\ of \numpages}
        \runningheadrule

        \newcommand{\examnum}{Integral definida}
        \begin{document}
        \begin{questions}
        \question p43e03a5 - Utiliza la regla de Barrow para calcular :

        \begin{multicols}{2}
        \begin{parts} \part[1] $\int\limits_{0}^{3} \left(3 x^{2} - 6\right)\, dx$  \begin{solution}   $9 \ (F(x)=x^{3} - 6 x)$   \end{solution} \part[1] $\int\limits_{1}^{2} \frac{1}{x}\, dx$  \begin{solution}   $\log{\left(2 \right)} \ (F(x)=\log{\left(x \right)})$   \end{solution} \part[1] $\int\limits_{0}^{1} \frac{5}{7 x^{2} + 7}\, dx$  \begin{solution}   $\frac{5 \pi}{28} \ (F(x)=\frac{5 \operatorname{atan}{\left(x \right)}}{7})$   \end{solution} \part[1] $\int\limits_{2}^{3} \frac{1}{x \log{\left(x \right)}}\, dx$  \begin{solution}   $\log{\left(\frac{\log{\left(3 \right)}}{\log{\left(2 \right)}} \right)} \ (F(x)=\log{\left(\log{\left(x \right)} \right)})$   \end{solution} \part[1] $\int\limits_{\frac{\pi}{2}}^{2 \pi} \sin^{5}{\left(x \right)} \cos{\left(x \right)}\, dx$  \begin{solution}   $- \frac{1}{6} \ (F(x)=\frac{\sin^{6}{\left(x \right)}}{6})$   \end{solution} \part[1] $\int\limits_{2}^{5} e^{x} x\, dx$  \begin{solution}   $- e^{2} + 4 e^{5} \ (F(x)=\left(x - 1\right) e^{x})$   \end{solution} \part[1] $\int\limits_{0}^{5} \begin{cases} x + 1 & \text{for}\: x < 1 \\3 - x & \text{for}\: x \leq 3 \\x - 3 & \text{otherwise} \end{cases}\, dx$  \begin{solution}   $\frac{11}{2} \ (F(x)=\begin{cases} \frac{x^{2}}{2} + x & \text{for}\: x < 1 \\- \frac{x^{2}}{2} + 3 x - 1 & \text{for}\: x \leq 3 \\\frac{x^{2}}{2} - 3 x + 8 & \text{otherwise} \end{cases})$   \end{solution} \part[1] $\int\limits_{-5}^{5} \left|{x}\right|\, dx$  \begin{solution}   $25 \ (F(x)=\int \left|{x}\right|\, dx)$   \end{solution} \part[1] $\int\limits_{0}^{\pi} \left|{x - 2}\right|\, dx$  \begin{solution}   $- 2 \pi + 4 + \frac{\pi^{2}}{2} \ (F(x)=\int \left|{x - 2}\right|\, dx)$   \end{solution}
        \end{parts}
        \end{multicols}
        \question p43e06a33 - Calcula:

        \begin{multicols}{1}
        \begin{parts} \part[1] El área del recinto limitado por la gráfica de la función 
$f(x) = x^3$, el eje de abscisas y las rectas x=-1
y x=1  \begin{solution}   Intervalos donde calcular la integral definida: \ $\left[-1, 0\right) \cup \left(0, 1\right] \to $ \\ Integrales definidas: \\ $\int\limits_{-1}^{0} x^{3}\, dx=\frac{1}{4} \ (F(x)=\frac{x^{4}}{4})$  \\ $\int\limits_{0}^{1} x^{3}\, dx=\frac{1}{4} \ (F(x)=\frac{x^{4}}{4})$  \\ Área total = $\frac{1}{2}$   \end{solution} \part[1] El área del recinto limitado por la gráfica de la función 
    $f(x) = -x^2 + 4$, el eje de abscisas y las rectas x=0
    y x=2  \begin{solution}   Intervalos donde calcular la integral definida: \ $\left[0, 2\right) \to $ \\ Integrales definidas: \\ $\int\limits_{0}^{2} \left(4 - x^{2}\right)\, dx=\frac{16}{3} \ (F(x)=- \frac{x^{3}}{3} + 4 x)$  \\ Área total = $\frac{16}{3}$   \end{solution} \part[1] El área del recinto limitado por la gráfica de $f(x)=x^4 -3x^3 -4x2 +12x$ y el eje OX  \begin{solution}   Intervalos donde calcular la integral definida: \ $\left(-\infty, -2\right) \cup \left(-2, 0\right) \cup \left(0, 2\right) \cup \left(2, 3\right) \cup \left(3, \infty\right) \to $ \\ Integrales definidas: \\ $\int\limits_{-\infty}^{-2} \left(x^{4} - 3 x^{3} - 4 x^{2} + 12 x\right)\, dx=\infty \ (F(x)=\frac{x^{5}}{5} - \frac{3 x^{4}}{4} - \frac{4 x^{3}}{3} + 6 x^{2})$  \\ $\int\limits_{-2}^{0} \left(x^{4} - 3 x^{3} - 4 x^{2} + 12 x\right)\, dx=\frac{244}{15} \ (F(x)=\frac{x^{5}}{5} - \frac{3 x^{4}}{4} - \frac{4 x^{3}}{3} + 6 x^{2})$  \\ $\int\limits_{0}^{2} \left(x^{4} - 3 x^{3} - 4 x^{2} + 12 x\right)\, dx=\frac{116}{15} \ (F(x)=\frac{x^{5}}{5} - \frac{3 x^{4}}{4} - \frac{4 x^{3}}{3} + 6 x^{2})$  \\ $\int\limits_{2}^{3} \left(x^{4} - 3 x^{3} - 4 x^{2} + 12 x\right)\, dx=\frac{113}{60} \ (F(x)=\frac{x^{5}}{5} - \frac{3 x^{4}}{4} - \frac{4 x^{3}}{3} + 6 x^{2})$  \\ $\int\limits_{3}^{\infty} \left(x^{4} - 3 x^{3} - 4 x^{2} + 12 x\right)\, dx=\infty \ (F(x)=\frac{x^{5}}{5} - \frac{3 x^{4}}{4} - \frac{4 x^{3}}{3} + 6 x^{2})$  \\ Área total = $\frac{1553}{60}$   \end{solution} \part[1] El área limitada por las gráficas de $f(x) = -x^2-4x+3 \land g(x) = x^2 -x-6$  \begin{solution}   Intervalos donde calcular la integral definida: \ $\left(-\infty, -3\right) \cup \left(-3, \frac{3}{2}\right) \cup \left(\frac{3}{2}, \infty\right) \to $ \\ Integrales definidas: \\ $\int\limits_{-\infty}^{-3} \left(\left(-2\right) x^{2} - 3 x + 9\right)\, dx=\infty \ (F(x)=- \frac{2 x^{3}}{3} - \frac{3 x^{2}}{2} + 9 x)$  \\ $\int\limits_{-3}^{\frac{3}{2}} \left(\left(-2\right) x^{2} - 3 x + 9\right)\, dx=\frac{243}{8} \ (F(x)=- \frac{2 x^{3}}{3} - \frac{3 x^{2}}{2} + 9 x)$  \\ $\int\limits_{\frac{3}{2}}^{\infty} \left(\left(-2\right) x^{2} - 3 x + 9\right)\, dx=\infty \ (F(x)=- \frac{2 x^{3}}{3} - \frac{3 x^{2}}{2} + 9 x)$  \\ Área total = $\frac{243}{8}$   \end{solution} \part[1] El área del recinto limitado por la recta $y = 3x + 2$, el eje OX y las rectas x = 1 y x = 3. Comprueba el
resultado por métodos geométricos  \begin{solution}   Intervalos donde calcular la integral definida: \ $\left[1, 3\right] \to $ \\ Integrales definidas: \\ $\int\limits_{1}^{3} \left(3 x + 2\right)\, dx=16 \ (F(x)=\frac{3 x^{2}}{2} + 2 x)$  \\ Área total = $16$   \end{solution} \part[1] El área limitada por las gráficas de las funciones 
    $f(x) = x+1 \land g(x) = x^2 +1$  \begin{solution}   Intervalos donde calcular la integral definida: \ $\left(-\infty, 0\right) \cup \left(0, 1\right) \cup \left(1, \infty\right) \to $ \\ Integrales definidas: \\ $\int\limits_{-\infty}^{0} \left(x - \left(x^{2} + 1\right) + 1\right)\, dx=\infty \ (F(x)=- \frac{x^{3}}{3} + \frac{x^{2}}{2})$  \\ $\int\limits_{0}^{1} \left(x - \left(x^{2} + 1\right) + 1\right)\, dx=\frac{1}{6} \ (F(x)=- \frac{x^{3}}{3} + \frac{x^{2}}{2})$  \\ $\int\limits_{1}^{\infty} \left(x - \left(x^{2} + 1\right) + 1\right)\, dx=\infty \ (F(x)=- \frac{x^{3}}{3} + \frac{x^{2}}{2})$  \\ Área total = $\frac{1}{6}$   \end{solution} \part[1] El área limitada por las gráficas de las funciones 
    $f(x) = 5x-9 \land g(x) = 3x^3 -21x^2 +47x-33$  \begin{solution}   Intervalos donde calcular la integral definida: \ $\left(-\infty, 1\right) \cup \left(1, 2\right) \cup \left(2, 4\right) \cup \left(4, \infty\right) \to $ \\ Integrales definidas: \\ $\int\limits_{-\infty}^{1} \left(5 x - \left(3 x^{3} - 21 x^{2} + 47 x - 33\right) - 9\right)\, dx=\infty \ (F(x)=- \frac{3 x^{4}}{4} + 7 x^{3} - 21 x^{2} + 24 x)$  \\ $\int\limits_{1}^{2} \left(5 x - \left(3 x^{3} - 21 x^{2} + 47 x - 33\right) - 9\right)\, dx=\frac{5}{4} \ (F(x)=- \frac{3 x^{4}}{4} + 7 x^{3} - 21 x^{2} + 24 x)$  \\ $\int\limits_{2}^{4} \left(5 x - \left(3 x^{3} - 21 x^{2} + 47 x - 33\right) - 9\right)\, dx=8 \ (F(x)=- \frac{3 x^{4}}{4} + 7 x^{3} - 21 x^{2} + 24 x)$  \\ $\int\limits_{4}^{\infty} \left(5 x - \left(3 x^{3} - 21 x^{2} + 47 x - 33\right) - 9\right)\, dx=\infty \ (F(x)=- \frac{3 x^{4}}{4} + 7 x^{3} - 21 x^{2} + 24 x)$  \\ Área total = $\frac{37}{4}$   \end{solution} \part[1] el área limitada por $f(x) = x^2 - 2x - 15$, el eje OX 
    y las rectas x = -4 y x = 7.  \begin{solution}   Intervalos donde calcular la integral definida: \ $\left[-4, -3\right) \cup \left(-3, 5\right) \cup \left(5, 7\right] \to $ \\ Integrales definidas: \\ $\int\limits_{-4}^{-3} \left(x^{2} - 2 x - 15\right)\, dx=\frac{13}{3} \ (F(x)=\frac{x^{3}}{3} - x^{2} - 15 x)$  \\ $\int\limits_{-3}^{5} \left(x^{2} - 2 x - 15\right)\, dx=\frac{256}{3} \ (F(x)=\frac{x^{3}}{3} - x^{2} - 15 x)$  \\ $\int\limits_{5}^{7} \left(x^{2} - 2 x - 15\right)\, dx=\frac{56}{3} \ (F(x)=\frac{x^{3}}{3} - x^{2} - 15 x)$  \\ Área total = $\frac{325}{3}$   \end{solution} \part[1] El área limitada por $f(x) = x^3 + 2x^2 - 5x -6$ 
    y el eje horizontal entre las abscisas -5 y $\frac{3}{2}$  \begin{solution}   Intervalos donde calcular la integral definida: \ $\left[-5, -3\right) \cup \left(-3, -1\right) \cup \left(-1, 1.5\right] \to $ \\ Integrales definidas: \\ $\int\limits_{-5}^{-3} \left(x^{3} + 2 x^{2} - 5 x - 6\right)\, dx=\frac{128}{3} \ (F(x)=\frac{x^{4}}{4} + \frac{2 x^{3}}{3} - \frac{5 x^{2}}{2} - 6 x)$  \\ $\int\limits_{-3}^{-1} \left(x^{3} + 2 x^{2} - 5 x - 6\right)\, dx=\frac{16}{3} \ (F(x)=\frac{x^{4}}{4} + \frac{2 x^{3}}{3} - \frac{5 x^{2}}{2} - 6 x)$  \\ $\int\limits_{-1}^{1.5} \left(x^{3} + 2 x^{2} - 5 x - 6\right)\, dx=14.1927083333333 \ (F(x)=\frac{x^{4}}{4} + \frac{2 x^{3}}{3} - \frac{5 x^{2}}{2} - 6 x)$  \\ Área total = $62.1927083333333$   \end{solution} \part[1] El área limitada por 
    $f(x) = x^3 + x^2 -10x + 8$ y el eje OX  \begin{solution}   Intervalos donde calcular la integral definida: \ $\left(-\infty, -4\right) \cup \left(-4, 1\right) \cup \left(1, 2\right) \cup \left(2, \infty\right) \to $ \\ Integrales definidas: \\ $\int\limits_{-\infty}^{-4} \left(x^{3} + x^{2} - 10 x + 8\right)\, dx=\infty \ (F(x)=\frac{x^{4}}{4} + \frac{x^{3}}{3} - 5 x^{2} + 8 x)$  \\ $\int\limits_{-4}^{1} \left(x^{3} + x^{2} - 10 x + 8\right)\, dx=\frac{875}{12} \ (F(x)=\frac{x^{4}}{4} + \frac{x^{3}}{3} - 5 x^{2} + 8 x)$  \\ $\int\limits_{1}^{2} \left(x^{3} + x^{2} - 10 x + 8\right)\, dx=\frac{11}{12} \ (F(x)=\frac{x^{4}}{4} + \frac{x^{3}}{3} - 5 x^{2} + 8 x)$  \\ $\int\limits_{2}^{\infty} \left(x^{3} + x^{2} - 10 x + 8\right)\, dx=\infty \ (F(x)=\frac{x^{4}}{4} + \frac{x^{3}}{3} - 5 x^{2} + 8 x)$  \\ Área total = $\frac{443}{6}$   \end{solution} \part[1] El área del recinto limitado por la 
    parábola $y = x^2 - 4x$ y el eje OX  \begin{solution}   Intervalos donde calcular la integral definida: \ $\left(-\infty, 0\right) \cup \left(0, 4\right) \cup \left(4, \infty\right) \to $ \\ Integrales definidas: \\ $\int\limits_{-\infty}^{0} \left(x^{2} - 4 x\right)\, dx=\infty \ (F(x)=\frac{x^{3}}{3} - 2 x^{2})$  \\ $\int\limits_{0}^{4} \left(x^{2} - 4 x\right)\, dx=\frac{32}{3} \ (F(x)=\frac{x^{3}}{3} - 2 x^{2})$  \\ $\int\limits_{4}^{\infty} \left(x^{2} - 4 x\right)\, dx=\infty \ (F(x)=\frac{x^{3}}{3} - 2 x^{2})$  \\ Área total = $\frac{32}{3}$   \end{solution} \part[1] El área del recinto limitado por la 
    parábola $y = 4x - x^2 $ y el eje OX  \begin{solution}   Intervalos donde calcular la integral definida: \ $\left(-\infty, 0\right) \cup \left(0, 4\right) \cup \left(4, \infty\right) \to $ \\ Integrales definidas: \\ $\int\limits_{-\infty}^{0} \left(- x^{2} + 4 x\right)\, dx=\infty \ (F(x)=- \frac{x^{3}}{3} + 2 x^{2})$  \\ $\int\limits_{0}^{4} \left(- x^{2} + 4 x\right)\, dx=\frac{32}{3} \ (F(x)=- \frac{x^{3}}{3} + 2 x^{2})$  \\ $\int\limits_{4}^{\infty} \left(- x^{2} + 4 x\right)\, dx=\infty \ (F(x)=- \frac{x^{3}}{3} + 2 x^{2})$  \\ Área total = $\frac{32}{3}$   \end{solution} \part[1] El área del recinto limitado por
    la recta $y = -2x$ y la parábola $y =-\frac{1}{2}x^2$  \begin{solution}   Intervalos donde calcular la integral definida: \ $\left(-\infty, 0\right) \cup \left(0, 4\right) \cup \left(4, \infty\right) \to $ \\ Integrales definidas: \\ $\int\limits_{-\infty}^{0} \left(\frac{x^{2}}{2} + \left(-2\right) x\right)\, dx=\infty \ (F(x)=\frac{x^{3}}{6} - x^{2})$  \\ $\int\limits_{0}^{4} \left(\frac{x^{2}}{2} + \left(-2\right) x\right)\, dx=\frac{16}{3} \ (F(x)=\frac{x^{3}}{6} - x^{2})$  \\ $\int\limits_{4}^{\infty} \left(\frac{x^{2}}{2} + \left(-2\right) x\right)\, dx=\infty \ (F(x)=\frac{x^{3}}{6} - x^{2})$  \\ Área total = $\frac{16}{3}$   \end{solution} \part[1] El área de la región limitada por la curva 
    y= x^3 - 8x^2 + 7x, el eje OX y las rectas x = 2 y x = 7  \begin{solution}   Intervalos donde calcular la integral definida: \ $\left[2, 7\right) \to $ \\ Integrales definidas: \\ $\int\limits_{2}^{7} \left(x^{3} - 8 x^{2} + 7 x\right)\, dx=\frac{1675}{12} \ (F(x)=\frac{x^{4}}{4} - \frac{8 x^{3}}{3} + \frac{7 x^{2}}{2})$  \\ Área total = $\frac{1675}{12}$   \end{solution} \part[1] El área de la región limitada por $f(x) = -e^x$ 
    el eje de abscisas y las rectas x = -1 y x = 2  \begin{solution}   Intervalos donde calcular la integral definida: \ $\left[-1, 2\right] \to $ \\ Integrales definidas: \\ $\int\limits_{-1}^{2} \left(- e^{x}\right)\, dx=- \frac{1}{e} + e^{2} \ (F(x)=- e^{x})$  \\ Área total = $- \frac{1}{e} + e^{2}$   \end{solution} \part[1] El área del recintor limitado por $f(x) = -ln \ x$ 
    el eje de abscisas y las rectas $x = e$ y $x = e^2$  \begin{solution}   Intervalos donde calcular la integral definida: \ $\left[e, e^{2}\right] \to $ \\ Integrales definidas: \\ $\int\limits_{e}^{e^{2}} \left(- \log{\left(x \right)}\right)\, dx=e^{2} \ (F(x)=- x \log{\left(x \right)} + x)$  \\ Área total = $e^{2}$   \end{solution}
        \end{parts}
        \end{multicols}
        
    \end{questions}
    \end{document}
    