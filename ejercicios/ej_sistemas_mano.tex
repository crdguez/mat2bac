
        \documentclass[spanish, 11pt]{exam}

        %These tell TeX which packages to use.
        \usepackage{array,epsfig}
        \usepackage{amsmath, textcomp}
        \usepackage{amsfonts}
        \usepackage{amssymb}
        \usepackage{amsxtra}
        \usepackage{amsthm}
        \usepackage{mathrsfs}
        \usepackage{color}
        \usepackage{multicol, xparse}
        \usepackage{verbatim}


        \usepackage[utf8]{inputenc}
        \usepackage[spanish]{babel}
        \usepackage{eurosym}

        \usepackage{graphicx}
        \graphicspath{{../img/}}
        \usepackage{pgf}



        \printanswers
        \nopointsinmargin
        \pointformat{}

        %Pagination stuff.
        %\setlength{\topmargin}{-.3 in}
        %\setlength{\oddsidemargin}{0in}
        %\setlength{\evensidemargin}{0in}
        %\setlength{\textheight}{9.in}
        %\setlength{\textwidth}{6.5in}
        %\pagestyle{empty}

        \let\multicolmulticols\multicols
        \let\endmulticolmulticols\endmulticols
        \RenewDocumentEnvironment{multicols}{mO{}}
         {%
          \ifnum#1=1
            #2%
          \else % More than 1 column
            \multicolmulticols{#1}[#2]
          \fi
         }
         {%
          \ifnum#1=1
          \else % More than 1 column
            \endmulticolmulticols
          \fi
         }
        \renewcommand{\solutiontitle}{\noindent\textbf{Sol:}\enspace}

        \newcommand{\samedir}{\mathbin{\!/\mkern-5mu/\!}}

        \newcommand{\class}{2º Bachillerato}
        \newcommand{\examdate}{\today}

        \newcommand{\tipo}{A}


        \newcommand{\timelimit}{50 minutos}



        \pagestyle{head}
        \firstpageheader{\includegraphics[width=0.2\columnwidth]{header_left}}{\textbf{Departamento de Matemáticas\linebreak \class}\linebreak \examnum}{\includegraphics[width=0.1\columnwidth]{header_right}}
        \runningheader{\class}{\examnum}{Página \thepage\ of \numpages}
        \runningheadrule

        \newcommand{\examnum}{Ejercicios de sistemas}
        \begin{document}
        \begin{questions}
        \question  - Ejercicios de sistemas con un parámetro:
        \begin{multicols}{1}
        \begin{parts} \part[1] Discutir y resolver el siguiente sistema con parámetro $k$: \\ $$\left\{ \begin{matrix}- 3 x + 2 y + 3 z = -2 \\ k y + 2 x - 5 z = -4 \\ x + y + 2 z = 2 \\ \end{matrix}\right.$$   \begin{solution}   \textbf{Discusión y resolución por Gauss:} Escalonando la matriz ampliada tenemos\\$A^*= \left(\begin{matrix}-3 & 2 & 3 & -2\\2 & k & -5 & -4\\1 & 1 & 2 & 2\end{matrix}\right) \thicksim \left(\begin{matrix}-3 & 2 & 3 & -2\\0 & k + \frac{4}{3} & -3 & - \frac{16}{3}\\0 & 0 & \frac{9 \left(k + 3\right)}{3 k + 4} & \frac{4 \left(k + 8\right)}{3 k + 4}\end{matrix}\right)$. \\  De los valores de la última fila podemos concluir:\begin{itemize}\item Si $k = -3 \to$ $$\left(\begin{matrix}-3 & 2 & 3 & -2\\0 & - \frac{5}{3} & -3 & - \frac{16}{3}\\0 & 0 & 0 & -4\end{matrix}\right)$$ La última fila es $0z=-4 \to $ S.I.\item si $k\neq [-3]  \to $ S.C.D.\begin{itemize}\item $\left(\begin{matrix}0 & 0 & \frac{9 \left(k + 3\right)}{3 k + 4} & \frac{4 \left(k + 8\right)}{3 k + 4}\end{matrix}\right) \to z = \frac{4 \left(k + 8\right)}{9 \left(k + 3\right)}$\end{itemize}\begin{itemize}\item $\left(\begin{matrix}0 & k + \frac{4}{3} & -3 & - \frac{16}{3}\end{matrix}\right) \to y = - \frac{4}{k + 3}$\end{itemize}\begin{itemize}\item $\left(\begin{matrix}-3 & 2 & 3 & -2\end{matrix}\right) \to x = \frac{2 \left(5 k + 13\right)}{9 \left(k + 3\right)}$\end{itemize}\end{itemize}  \textbf{Por rangos y determinantes:} \\$\left|A\right|=- 9 k - 27 \to \left|A\right|=0 \quad si \quad k = \left[ -3\right]$\begin{itemize}\item Si $k=-3 \to rg(A)=2 \land rg(A^*)=3 \to $ S.I.\item Si $k\neq[-3] \to rg(A)=3 \land rg(A^*)=3 \to $ S.C.D. $\to$ \\ Por Cramer: \begin{itemize}\item $x=\frac{\left|\begin{matrix}-2 & 2 & 3\\-4 & k & -5\\2 & 1 & 2\end{matrix}\right|}{- 9 k - 27}=\frac{- 10 k - 26}{- 9 k - 27}=\frac{2 \left(5 k + 13\right)}{9 \left(k + 3\right)}$\item $y=\frac{\left|\begin{matrix}-3 & -2 & 3\\2 & -4 & -5\\1 & 2 & 2\end{matrix}\right|}{- 9 k - 27}=\frac{36}{- 9 k - 27}=- \frac{4}{k + 3}$\item $z=\frac{\left|\begin{matrix}-3 & 2 & -2\\2 & k & -4\\1 & 1 & 2\end{matrix}\right|}{- 9 k - 27}=\frac{- 4 k - 32}{- 9 k - 27}=\frac{4 \left(k + 8\right)}{9 \left(k + 3\right)}$\end{itemize}\end{itemize}   \end{solution} \part[1] Discutir y resolver el siguiente sistema con parámetro $k$: \\ $$\left\{ \begin{matrix}k x + y + z = k \\ k y + x + z = k \\ k z + x + y = k \\ \end{matrix}\right.$$   \begin{solution}   \textbf{Discusión y resolución por Gauss:} Escalonando la matriz ampliada tenemos\\$A^*= \left(\begin{matrix}k & 1 & 1 & k\\1 & k & 1 & k\\1 & 1 & k & k\end{matrix}\right) \thicksim \left(\begin{matrix}1 & k & 1 & k\\0 & 1 - k^{2} & 1 - k & k \left(1 - k\right)\\0 & 0 & \frac{k^{2} + k - 2}{k + 1} & \frac{k \left(k - 1\right)}{k + 1}\end{matrix}\right)$. \\  De los valores de la última fila podemos concluir:\begin{itemize}\item Si $k = -2 \to$ $$\left(\begin{matrix}1 & -2 & 1 & -2\\0 & -3 & 3 & -6\\0 & 0 & 0 & -6\end{matrix}\right)$$ La última fila es $0z=-6 \to $ S.I.\item Si $k = 1 \to$ $$\left(\begin{matrix}1 & 1 & 1 & 1\\0 & 0 & 0 & 0\\0 & 0 & 0 & 0\end{matrix}\right)$$ La última fila es $0z=0 \to $ S.C.I\begin{itemize}\item $\left(\begin{matrix}0 & 0 & 0 & 0\end{matrix}\right) \to z = \lambda$\end{itemize}\begin{itemize}\item $\left(\begin{matrix}0 & 0 & 0 & 0\end{matrix}\right) \to y = \mu$\end{itemize}\begin{itemize}\item $\left(\begin{matrix}1 & 1 & 1 & 1\end{matrix}\right) \to x = - \mu - \lambda + 1$\end{itemize}\item si $k\neq [-2, 1]  \to $ S.C.D.\begin{itemize}\item $\left(\begin{matrix}0 & 0 & \frac{k^{2} + k - 2}{k + 1} & \frac{k \left(k - 1\right)}{k + 1}\end{matrix}\right) \to z = \frac{k}{k + 2}$\end{itemize}\begin{itemize}\item $\left(\begin{matrix}0 & 1 - k^{2} & 1 - k & k \left(1 - k\right)\end{matrix}\right) \to y = \frac{k}{k + 2}$\end{itemize}\begin{itemize}\item $\left(\begin{matrix}1 & k & 1 & k\end{matrix}\right) \to x = \frac{k}{k + 2}$\end{itemize}\end{itemize}  \textbf{Por rangos y determinantes:} \\$\left|A\right|=k^{3} - 3 k + 2 \to \left|A\right|=0 \quad si \quad k = \left[ -2, \  1\right]$\begin{itemize}\item Si $k=-2 \to rg(A)=2 \land rg(A^*)=3 \to $ S.I.\item Si $k=1 \to rg(A)=1 \land rg(A^*)=1 \to $ S.C.I. $\to$ solo se puede resolver por Gauss, (ver más arriba)\item Si $k\neq[-2, 1] \to rg(A)=3 \land rg(A^*)=3 \to $ S.C.D. $\to$ \\ Por Cramer: \begin{itemize}\item $x=\frac{\left|\begin{matrix}k & 1 & 1\\k & k & 1\\k & 1 & k\end{matrix}\right|}{k^{3} - 3 k + 2}=\frac{k \left(k^{2} - 2 k + 1\right)}{k^{3} - 3 k + 2}=\frac{k}{k + 2}$\item $y=\frac{\left|\begin{matrix}k & k & 1\\1 & k & 1\\1 & k & k\end{matrix}\right|}{k^{3} - 3 k + 2}=\frac{k \left(k^{2} - 2 k + 1\right)}{k^{3} - 3 k + 2}=\frac{k}{k + 2}$\item $z=\frac{\left|\begin{matrix}k & 1 & k\\1 & k & k\\1 & 1 & k\end{matrix}\right|}{k^{3} - 3 k + 2}=\frac{k \left(k^{2} - 2 k + 1\right)}{k^{3} - 3 k + 2}=\frac{k}{k + 2}$\end{itemize}\end{itemize}   \end{solution} \part[1] Discutir y resolver el siguiente sistema con parámetro $k$: \\ $$\left\{ \begin{matrix}x + y + z = k + 2 \\ - k y + x + z = 1 \\ k x + y + z = 4 \\ \end{matrix}\right.$$   \begin{solution}   \textbf{Discusión y resolución por Gauss:} Escalonando la matriz ampliada tenemos\\$A^*= \left(\begin{matrix}1 & 1 & 1 & k + 2\\1 & - k & 1 & 1\\k & 1 & 1 & 4\end{matrix}\right) \thicksim \left(\begin{matrix}1 & 1 & 1 & k + 2\\0 & - k - 1 & 0 & - k - 1\\0 & 0 & 1 - k & - k^{2} - k + 3\end{matrix}\right)$. \\  De los valores de la última fila podemos concluir:\begin{itemize}\item Si $k = 1 \to$ $$\left(\begin{matrix}1 & 1 & 1 & 3\\0 & -2 & 0 & -2\\0 & 0 & 0 & 1\end{matrix}\right)$$ La última fila es $0z=1 \to $ S.I.\item si $k\neq [1]  \to $ S.C.D.\begin{itemize}\item $\left(\begin{matrix}0 & 0 & 1 - k & - k^{2} - k + 3\end{matrix}\right) \to z = \frac{k^{2} + k - 3}{k - 1}$\end{itemize}\begin{itemize}\item $\left(\begin{matrix}0 & - k - 1 & 0 & - k - 1\end{matrix}\right) \to y = 1$\end{itemize}\begin{itemize}\item $\left(\begin{matrix}1 & 1 & 1 & k + 2\end{matrix}\right) \to x = \frac{2 - k}{k - 1}$\end{itemize}\end{itemize}  \textbf{Por rangos y determinantes:} \\$\left|A\right|=k^{2} - 1 \to \left|A\right|=0 \quad si \quad k = \left[ -1, \  1\right]$\begin{itemize}\item Si $k=-1 \to rg(A)=2 \land rg(A^*)=2 \to $ S.C.I. $\to$ solo se puede resolver por Gauss, (ver más arriba)\item Si $k=1 \to rg(A)=2 \land rg(A^*)=3 \to $ S.I.\item Si $k\neq[-1, 1] \to rg(A)=3 \land rg(A^*)=3 \to $ S.C.D. $\to$ \\ Por Cramer: \begin{itemize}\item $x=\frac{\left|\begin{matrix}k + 2 & 1 & 1\\1 & - k & 1\\4 & 1 & 1\end{matrix}\right|}{k^{2} - 1}=\frac{- k^{2} + k + 2}{k^{2} - 1}=\frac{2 - k}{k - 1}$\item $y=\frac{\left|\begin{matrix}1 & k + 2 & 1\\1 & 1 & 1\\k & 4 & 1\end{matrix}\right|}{k^{2} - 1}=\frac{k^{2} - 1}{k^{2} - 1}=1$\item $z=\frac{\left|\begin{matrix}1 & 1 & k + 2\\1 & - k & 1\\k & 1 & 4\end{matrix}\right|}{k^{2} - 1}=\frac{k^{2} \left(k + 2\right) - 2 k - 3}{k^{2} - 1}=\frac{k^{2} + k - 3}{k - 1}$\end{itemize}\end{itemize}   \end{solution} \part[1] Discutir y resolver el siguiente sistema con parámetro $k$: \\ $$\left\{ \begin{matrix}k x + k z + y \left(k^{2} + 1\right) = k \\ k y + x + z = 0 \\ k^{2} z + x + y \left(k + 1\right) = 2 k - 1 \\ \end{matrix}\right.$$   \begin{solution}   \textbf{Discusión y resolución por Gauss:} Escalonando la matriz ampliada tenemos\\$A^*= \left(\begin{matrix}k & k^{2} + 1 & k & k\\1 & k & 1 & 0\\1 & k + 1 & k^{2} & 2 k - 1\end{matrix}\right) \thicksim \left(\begin{matrix}1 & k & 1 & 0\\0 & 1 & 0 & k\\0 & 0 & k^{2} - 1 & k - 1\end{matrix}\right)$. \\  De los valores de la última fila podemos concluir:\begin{itemize}\item Si $k = -1 \to$ $$\left(\begin{matrix}1 & -1 & 1 & 0\\0 & 1 & 0 & -1\\0 & 0 & 0 & -2\end{matrix}\right)$$ La última fila es $0z=-2 \to $ S.I.\item Si $k = 1 \to$ $$\left(\begin{matrix}1 & 1 & 1 & 0\\0 & 1 & 0 & 1\\0 & 0 & 0 & 0\end{matrix}\right)$$ La última fila es $0z=0 \to $ S.C.I\begin{itemize}\item $\left(\begin{matrix}0 & 0 & 0 & 0\end{matrix}\right) \to z = \lambda$\end{itemize}\begin{itemize}\item $\left(\begin{matrix}0 & 1 & 0 & 1\end{matrix}\right) \to y = 1$\end{itemize}\begin{itemize}\item $\left(\begin{matrix}1 & 1 & 1 & 0\end{matrix}\right) \to x = - \lambda - 1$\end{itemize}\item si $k\neq [-1, 1]  \to $ S.C.D.\begin{itemize}\item $\left(\begin{matrix}0 & 0 & k^{2} - 1 & k - 1\end{matrix}\right) \to z = \frac{1}{k + 1}$\end{itemize}\begin{itemize}\item $\left(\begin{matrix}0 & 1 & 0 & k\end{matrix}\right) \to y = k$\end{itemize}\begin{itemize}\item $\left(\begin{matrix}1 & k & 1 & 0\end{matrix}\right) \to x = - \frac{k^{3} + k^{2} + 1}{k + 1}$\end{itemize}\end{itemize}  \textbf{Por rangos y determinantes:} \\$\left|A\right|=k^{4} - k^{2} \left(k^{2} + 1\right) + 1 \to \left|A\right|=0 \quad si \quad k = \left[ -1, \  1\right]$\begin{itemize}\item Si $k=-1 \to rg(A)=2 \land rg(A^*)=3 \to $ S.I.\item Si $k=1 \to rg(A)=2 \land rg(A^*)=2 \to $ S.C.I. $\to$ solo se puede resolver por Gauss, (ver más arriba)\item Si $k\neq[-1, 1] \to rg(A)=3 \land rg(A^*)=3 \to $ S.C.D. $\to$ \\ Por Cramer: \begin{itemize}\item $x=\frac{\left|\begin{matrix}k & k^{2} + 1 & k\\0 & k & 1\\2 k - 1 & k + 1 & k^{2}\end{matrix}\right|}{1 - k^{2}}=\frac{k^{4} - k^{2} + k - 1}{1 - k^{2}}=- \frac{k^{3} + k^{2} + 1}{k + 1}$\item $y=\frac{\left|\begin{matrix}k & k & k\\1 & 0 & 1\\1 & 2 k - 1 & k^{2}\end{matrix}\right|}{1 - k^{2}}=\frac{- k^{3} + k}{1 - k^{2}}=k$\item $z=\frac{\left|\begin{matrix}k & k^{2} + 1 & k\\1 & k & 0\\1 & k + 1 & 2 k - 1\end{matrix}\right|}{1 - k^{2}}=\frac{1 - k}{1 - k^{2}}=\frac{1}{k + 1}$\end{itemize}\end{itemize}   \end{solution} \part[1] Discutir y resolver el siguiente sistema con parámetro $k$: \\ $$\left\{ \begin{matrix}2 x - 5 y + 3 z = 0 \\ x - y + z = 0 \\ k y + 3 x + z = 0 \\ \end{matrix}\right.$$   \begin{solution}   \textbf{Discusión y resolución por Gauss:} Escalonando la matriz ampliada tenemos\\$A^*= \left(\begin{matrix}2 & -5 & 3 & 0\\1 & -1 & 1 & 0\\3 & k & 1 & 0\end{matrix}\right) \thicksim \left(\begin{matrix}2 & -5 & 3 & 0\\0 & \frac{3}{2} & - \frac{1}{2} & 0\\0 & 0 & \frac{k}{3} - 1 & 0\end{matrix}\right)$. \\  De los valores de la última fila podemos concluir:\begin{itemize}\item Si $k = 3 \to$ $$\left(\begin{matrix}2 & -5 & 3 & 0\\0 & \frac{3}{2} & - \frac{1}{2} & 0\\0 & 0 & 0 & 0\end{matrix}\right)$$ La última fila es $0z=0 \to $ S.C.I\begin{itemize}\item $\left(\begin{matrix}0 & 0 & 0 & 0\end{matrix}\right) \to z = \lambda$\end{itemize}\begin{itemize}\item $\left(\begin{matrix}0 & \frac{3}{2} & - \frac{1}{2} & 0\end{matrix}\right) \to y = \frac{\lambda}{3}$\end{itemize}\begin{itemize}\item $\left(\begin{matrix}2 & -5 & 3 & 0\end{matrix}\right) \to x = - \frac{2 \lambda}{3}$\end{itemize}\item si $k\neq [3]  \to $ S.C.D.\begin{itemize}\item $\left(\begin{matrix}0 & 0 & \frac{k}{3} - 1 & 0\end{matrix}\right) \to z = 0$\end{itemize}\begin{itemize}\item $\left(\begin{matrix}0 & \frac{3}{2} & - \frac{1}{2} & 0\end{matrix}\right) \to y = 0$\end{itemize}\begin{itemize}\item $\left(\begin{matrix}2 & -5 & 3 & 0\end{matrix}\right) \to x = 0$\end{itemize}\end{itemize}  \textbf{Por rangos y determinantes:} \\$\left|A\right|=k - 3 \to \left|A\right|=0 \quad si \quad k = \left[ 3\right]$\begin{itemize}\item Si $k=3 \to rg(A)=2 \land rg(A^*)=2 \to $ S.C.I. $\to$ solo se puede resolver por Gauss, (ver más arriba)\item Si $k\neq[3] \to rg(A)=3 \land rg(A^*)=3 \to $ S.C.D. $\to$ \\ Por Cramer: \begin{itemize}\item $x=\frac{\left|\begin{matrix}0 & -5 & 3\\0 & -1 & 1\\0 & k & 1\end{matrix}\right|}{k - 3}=\frac{0}{k - 3}=0$\item $y=\frac{\left|\begin{matrix}2 & 0 & 3\\1 & 0 & 1\\3 & 0 & 1\end{matrix}\right|}{k - 3}=\frac{0}{k - 3}=0$\item $z=\frac{\left|\begin{matrix}2 & -5 & 0\\1 & -1 & 0\\3 & k & 0\end{matrix}\right|}{k - 3}=\frac{0}{k - 3}=0$\end{itemize}\end{itemize}   \end{solution} \part[1] Discutir y resolver el siguiente sistema con parámetro $k$: \\ $$\left\{ \begin{matrix}k y + x - z = 0 \\ 12 x - 3 y - 2 z = 0 \\ x - 2 y + z = 0 \\ \end{matrix}\right.$$   \begin{solution}   \textbf{Discusión y resolución por Gauss:} Escalonando la matriz ampliada tenemos\\$A^*= \left(\begin{matrix}1 & k & -1 & 0\\12 & -3 & -2 & 0\\1 & -2 & 1 & 0\end{matrix}\right) \thicksim \left(\begin{matrix}1 & k & -1 & 0\\0 & - 12 k - 3 & 10 & 0\\0 & 0 & \frac{14 \left(k - 1\right)}{3 \left(4 k + 1\right)} & 0\end{matrix}\right)$. \\  De los valores de la última fila podemos concluir:\begin{itemize}\item Si $k = 1 \to$ $$\left(\begin{matrix}1 & 1 & -1 & 0\\0 & -15 & 10 & 0\\0 & 0 & 0 & 0\end{matrix}\right)$$ La última fila es $0z=0 \to $ S.C.I\begin{itemize}\item $\left(\begin{matrix}0 & 0 & 0 & 0\end{matrix}\right) \to z = \lambda$\end{itemize}\begin{itemize}\item $\left(\begin{matrix}0 & -15 & 10 & 0\end{matrix}\right) \to y = \frac{2 \lambda}{3}$\end{itemize}\begin{itemize}\item $\left(\begin{matrix}1 & 1 & -1 & 0\end{matrix}\right) \to x = \frac{\lambda}{3}$\end{itemize}\item si $k\neq [1]  \to $ S.C.D.\begin{itemize}\item $\left(\begin{matrix}0 & 0 & \frac{14 \left(k - 1\right)}{3 \left(4 k + 1\right)} & 0\end{matrix}\right) \to z = 0$\end{itemize}\begin{itemize}\item $\left(\begin{matrix}0 & - 12 k - 3 & 10 & 0\end{matrix}\right) \to y = 0$\end{itemize}\begin{itemize}\item $\left(\begin{matrix}1 & k & -1 & 0\end{matrix}\right) \to x = 0$\end{itemize}\end{itemize}  \textbf{Por rangos y determinantes:} \\$\left|A\right|=14 - 14 k \to \left|A\right|=0 \quad si \quad k = \left[ 1\right]$\begin{itemize}\item Si $k=1 \to rg(A)=2 \land rg(A^*)=2 \to $ S.C.I. $\to$ solo se puede resolver por Gauss, (ver más arriba)\item Si $k\neq[1] \to rg(A)=3 \land rg(A^*)=3 \to $ S.C.D. $\to$ \\ Por Cramer: \begin{itemize}\item $x=\frac{\left|\begin{matrix}0 & k & -1\\0 & -3 & -2\\0 & -2 & 1\end{matrix}\right|}{14 - 14 k}=\frac{0}{14 - 14 k}=0$\item $y=\frac{\left|\begin{matrix}1 & 0 & -1\\12 & 0 & -2\\1 & 0 & 1\end{matrix}\right|}{14 - 14 k}=\frac{0}{14 - 14 k}=0$\item $z=\frac{\left|\begin{matrix}1 & k & 0\\12 & -3 & 0\\1 & -2 & 0\end{matrix}\right|}{14 - 14 k}=\frac{0}{14 - 14 k}=0$\end{itemize}\end{itemize}   \end{solution} \part[1] Discutir y resolver el siguiente sistema con parámetro $k$: \\ $$\left\{ \begin{matrix}x + y + 2 z = 0 \\ k x + y - z = k - 2 \\ k y + 3 x + z = k - 2 \\ \end{matrix}\right.$$   \begin{solution}   \textbf{Discusión y resolución por Gauss:} Escalonando la matriz ampliada tenemos\\$A^*= \left(\begin{matrix}1 & 1 & 2 & 0\\k & 1 & -1 & k - 2\\3 & k & 1 & k - 2\end{matrix}\right) \thicksim \left(\begin{matrix}1 & 1 & 2 & 0\\0 & 1 - k & - 2 k - 1 & k - 2\\0 & 0 & \frac{8 - 2 k^{2}}{k - 1} & \frac{2 \left(k - 2\right)^{2}}{k - 1}\end{matrix}\right)$. \\  De los valores de la última fila podemos concluir:\begin{itemize}\item Si $k = -2 \to$ $$\left(\begin{matrix}1 & 1 & 2 & 0\\0 & 3 & 3 & -4\\0 & 0 & 0 & - \frac{32}{3}\end{matrix}\right)$$ La última fila es $0z=-32/3 \to $ S.I.\item Si $k = 2 \to$ $$\left(\begin{matrix}1 & 1 & 2 & 0\\0 & -1 & -5 & 0\\0 & 0 & 0 & 0\end{matrix}\right)$$ La última fila es $0z=0 \to $ S.C.I\begin{itemize}\item $\left(\begin{matrix}0 & 0 & 0 & 0\end{matrix}\right) \to z = \lambda$\end{itemize}\begin{itemize}\item $\left(\begin{matrix}0 & -1 & -5 & 0\end{matrix}\right) \to y = - 5 \lambda$\end{itemize}\begin{itemize}\item $\left(\begin{matrix}1 & 1 & 2 & 0\end{matrix}\right) \to x = 3 \lambda$\end{itemize}\item si $k\neq [-2, 2]  \to $ S.C.D.\begin{itemize}\item $\left(\begin{matrix}0 & 0 & \frac{8 - 2 k^{2}}{k - 1} & \frac{2 \left(k - 2\right)^{2}}{k - 1}\end{matrix}\right) \to z = \frac{2 - k}{k + 2}$\end{itemize}\begin{itemize}\item $\left(\begin{matrix}0 & 1 - k & - 2 k - 1 & k - 2\end{matrix}\right) \to y = \frac{k - 2}{k + 2}$\end{itemize}\begin{itemize}\item $\left(\begin{matrix}1 & 1 & 2 & 0\end{matrix}\right) \to x = \frac{k - 2}{k + 2}$\end{itemize}\end{itemize}  \textbf{Por rangos y determinantes:} \\$\left|A\right|=2 k^{2} - 8 \to \left|A\right|=0 \quad si \quad k = \left[ -2, \  2\right]$\begin{itemize}\item Si $k=-2 \to rg(A)=2 \land rg(A^*)=3 \to $ S.I.\item Si $k=2 \to rg(A)=2 \land rg(A^*)=2 \to $ S.C.I. $\to$ solo se puede resolver por Gauss, (ver más arriba)\item Si $k\neq[-2, 2] \to rg(A)=3 \land rg(A^*)=3 \to $ S.C.D. $\to$ \\ Por Cramer: \begin{itemize}\item $x=\frac{\left|\begin{matrix}0 & 1 & 2\\k - 2 & 1 & -1\\k - 2 & k & 1\end{matrix}\right|}{2 k^{2} - 8}=\frac{2 k^{2} - 8 k + 8}{2 k^{2} - 8}=\frac{k - 2}{k + 2}$\item $y=\frac{\left|\begin{matrix}1 & 0 & 2\\k & k - 2 & -1\\3 & k - 2 & 1\end{matrix}\right|}{2 k^{2} - 8}=\frac{2 k^{2} - 8 k + 8}{2 k^{2} - 8}=\frac{k - 2}{k + 2}$\item $z=\frac{\left|\begin{matrix}1 & 1 & 0\\k & 1 & k - 2\\3 & k & k - 2\end{matrix}\right|}{2 k^{2} - 8}=\frac{- 2 k^{2} + 8 k - 8}{2 k^{2} - 8}=\frac{2 - k}{k + 2}$\end{itemize}\end{itemize}   \end{solution}
        \end{parts}
        \end{multicols}
        
    \end{questions}
    \end{document}
    