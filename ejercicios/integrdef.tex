
        \documentclass[spanish, 11pt]{exam}

        %These tell TeX which packages to use.
        \usepackage{array,epsfig}
        \usepackage{amsmath, textcomp}
        \usepackage{amsfonts}
        \usepackage{amssymb}
        \usepackage{amsxtra}
        \usepackage{amsthm}
        \usepackage{mathrsfs}
        \usepackage{color}
        \usepackage{multicol, xparse}
        \usepackage{verbatim}


        \usepackage[utf8]{inputenc}
        \usepackage[spanish]{babel}
        \usepackage{eurosym}

        \usepackage{graphicx}
        \graphicspath{{../img/}}
        \usepackage{pgf}



        \printanswers
        \nopointsinmargin
        \pointformat{}

        %Pagination stuff.
        %\setlength{\topmargin}{-.3 in}
        %\setlength{\oddsidemargin}{0in}
        %\setlength{\evensidemargin}{0in}
        %\setlength{\textheight}{9.in}
        %\setlength{\textwidth}{6.5in}
        %\pagestyle{empty}

        \let\multicolmulticols\multicols
        \let\endmulticolmulticols\endmulticols
        \RenewDocumentEnvironment{multicols}{mO{}}
         {%
          \ifnum#1=1
            #2%
          \else % More than 1 column
            \multicolmulticols{#1}[#2]
          \fi
         }
         {%
          \ifnum#1=1
          \else % More than 1 column
            \endmulticolmulticols
          \fi
         }
        \renewcommand{\solutiontitle}{\noindent\textbf{Sol:}\enspace}

        \newcommand{\samedir}{\mathbin{\!/\mkern-5mu/\!}}

        \newcommand{\class}{2º Bachillerato}
        \newcommand{\examdate}{\today}

        \newcommand{\tipo}{A}


        \newcommand{\timelimit}{50 minutos}



        \pagestyle{head}
        \firstpageheader{\includegraphics[width=0.2\columnwidth]{header_left}}{\textbf{Departamento de Matemáticas\linebreak \class}\linebreak \examnum}{\includegraphics[width=0.1\columnwidth]{header_right}}
        \runningheader{\class}{\examnum}{Página \thepage\ of \numpages}
        \runningheadrule

        \newcommand{\examnum}{Integral definida}
        \begin{document}
        \begin{questions}
        \question p43e03 - Utiliza la regla de Barrow para calcular :

        \begin{multicols}{2}
        \begin{parts} \part[1] $\int_{0}^{3} \left(3 x^{2} - 6\right)\, dx$  \begin{solution}   $9 \ (F(x)=x^{3} - 6 x)$   \end{solution} \part[1] $\int_{1}^{2} \frac{1}{x}\, dx$  \begin{solution}   $\log{\left (2 \right )} \ (F(x)=\log{\left (x \right )})$   \end{solution} \part[1] $\int_{0}^{1} \frac{5}{7 x^{2} + 7}\, dx$  \begin{solution}   $\frac{5 \pi}{28} \ (F(x)=\frac{5 \operatorname{atan}{\left (x \right )}}{7})$   \end{solution} \part[1] $\int_{2}^{3} \frac{1}{x \log{\left (x \right )}}\, dx$  \begin{solution}   $\log{\left (\frac{\log{\left (3 \right )}}{\log{\left (2 \right )}} \right )} \ (F(x)=\log{\left (\log{\left (x \right )} \right )})$   \end{solution} \part[1] $\int_{\frac{\pi}{2}}^{2 \pi} \sin^{5}{\left (x \right )} \cos{\left (x \right )}\, dx$  \begin{solution}   $- \frac{1}{6} \ (F(x)=\frac{\sin^{6}{\left (x \right )}}{6})$   \end{solution} \part[1] $\int_{2}^{5} e^{x} x\, dx$  \begin{solution}   $- e^{2} + 4 e^{5} \ (F(x)=\left(x - 1\right) e^{x})$   \end{solution}
        \end{parts}
        \end{multicols}
        \question p43e06 - Calcula:

        \begin{multicols}{2}
        \begin{parts} \part[1] el área del recinto limitado por la gráfica de la función 
$f(x) = -x^2 + 4$, el eje de abscisas y las rectas x=0
y x=2  \begin{solution}   $\int_{0}^{2} \left(- x^{2} + 4\right)\, dx=$$\frac{16}{3} \ (F(x)=- \frac{x^{3}}{3} + 4 x)$   \end{solution}
        \end{parts}
        \end{multicols}
        
    \end{questions}
    \end{document}
    