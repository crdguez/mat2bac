
        \documentclass[spanish, 11pt]{exam}

        %These tell TeX which packages to use.
        \usepackage{array,epsfig}
        \usepackage{amsmath, textcomp}
        \usepackage{amsfonts}
        \usepackage{amssymb}
        \usepackage{amsxtra}
        \usepackage{amsthm}
        \usepackage{mathrsfs}
        \usepackage{color}
        \usepackage{multicol, xparse}
        \usepackage{verbatim}


        \usepackage[utf8]{inputenc}
        \usepackage[spanish]{babel}
        \usepackage{eurosym}

        \usepackage{graphicx}
        \graphicspath{{../img/}}
        \usepackage{pgf}



        \printanswers
        \nopointsinmargin
        \pointformat{}

        %Pagination stuff.
        %\setlength{\topmargin}{-.3 in}
        %\setlength{\oddsidemargin}{0in}
        %\setlength{\evensidemargin}{0in}
        %\setlength{\textheight}{9.in}
        %\setlength{\textwidth}{6.5in}
        %\pagestyle{empty}

        \let\multicolmulticols\multicols
        \let\endmulticolmulticols\endmulticols
        \RenewDocumentEnvironment{multicols}{mO{}}
         {%
          \ifnum#1=1
            #2%
          \else % More than 1 column
            \multicolmulticols{#1}[#2]
          \fi
         }
         {%
          \ifnum#1=1
          \else % More than 1 column
            \endmulticolmulticols
          \fi
         }
        \renewcommand{\solutiontitle}{\noindent\textbf{Sol:}\enspace}

        \newcommand{\samedir}{\mathbin{\!/\mkern-5mu/\!}}

        \newcommand{\class}{2º Bachillerato}
        \newcommand{\examdate}{\today}

        \newcommand{\tipo}{A}


        \newcommand{\timelimit}{50 minutos}



        \pagestyle{head}
        \firstpageheader{\includegraphics[width=0.2\columnwidth]{header_left}}{\textbf{Departamento de Matemáticas\linebreak \class}\linebreak \examnum}{\includegraphics[width=0.1\columnwidth]{header_right}}
        \runningheader{\class}{\examnum}{Página \thepage\ of \numpages}
        \runningheadrule

        \newcommand{\examnum}{Optimización de funciones}
        \begin{document}
        \begin{questions}
        \question p31e13a38 - Problemas de optimización
        \begin{multicols}{1}
        \begin{parts} \part[1] 13. Descomponer el número 48 en dos sumandos tales que el quíntuplo del cuadrado del primero más el séxtuplo del
cuadrado del segundo sea mínimo.  \begin{solution}   \\ Función a optimizar y derivadas: \\ $\left[ f(x), \  5 x^{2} + 6 \left(48 - x\right)^{2}\right]$ \\ $\left[ f'(x), \  22 x - 576\right]$ \\ $\left[ f''(x), \  22\right]$ \\Extremos relativos: \\ $\left[ MINIMO, \  x, \  \frac{288}{11}\right]$ \\\\ \\ \\  \\  \\   \end{solution} \part[1] 14. Halla el número positivo cuya suma, con 4 veces su recíproco, sea mínima.  \begin{solution}   \\ Función a optimizar y derivadas: \\ $\left[ f(x), \  x + \frac{4}{x}\right]$ \\ $\left[ f'(x), \  1 - \frac{4}{x^{2}}\right]$ \\ $\left[ f''(x), \  \frac{8}{x^{3}}\right]$ \\Extremos relativos: \\ $\left[ MAXIMO, \  x, \  -2\right]$ \\ $\left[ MINIMO, \  x, \  2\right]$ \\\\ \\ \\  \\  \\   \end{solution} \part[1] 15. Halla las dimensiones del rectángulo de área máxima inscrito en una circunferencia de 20 cm de radio.  \begin{solution}   \\ Función a optimizar y derivadas: \\ $\left[ f(x), \  x \sqrt{- x^{2} + 40^{2}}\right]$ \\ $\left[ f'(x), \  \frac{2 \left(800 - x^{2}\right)}{\sqrt{1600 - x^{2}}}\right]$ \\ $\left[ f''(x), \  \frac{2 x \left(x^{2} - 2400\right)}{\left(1600 - x^{2}\right)^{\frac{3}{2}}}\right]$ \\Extremos relativos: \\ $\left[ MINIMO, \  x, \  - 20 \sqrt{2}\right]$ \\ $\left[ MAXIMO, \  x, \  20 \sqrt{2}\right]$ \\\\ \\ \\  \\  \\   \end{solution} \part[1] 16. Se considera una ventana rectangular en la que el lado superior ha sido sustituido por un triángulo equilátero. Sabiendo
que el perímetro de la ventana es de 6,6 m, halla sus dimensiones para que su superficie sea máxima.  \begin{solution}   \\ Función a optimizar y derivadas: \\ $\left[ f(x), \  \frac{x \left(6.6 - 3 x\right)}{2} + \frac{x \sqrt{x^{2} - \frac{x^{2}}{4}}}{2}\right]$ \\ $\left[ f'(x), \  \frac{0.5 \sqrt{3} x^{2}}{\sqrt{x^{2}}} - 3.0 x + 3.3\right]$ \\ $\left[ f''(x), \  -3 - \frac{\sqrt{3} \sqrt{x^{2}}}{4 x} + \frac{3 \sqrt{3} \left(x^{2}\right)^{\frac{3}{2}}}{4 x^{3}}\right]$ \\Extremos relativos: \\ $\left[ MAXIMO, \  x, \  1.54641016151378\right]$ \\\\ \\ \\  \\  \\   \end{solution} \part[1] 18. Una hoja de papel debe contener 18 cm 2 de texto impreso, márgenes superior e inferior de 2 cm de altura y márgenes
laterales de 1 cm de anchura. Obtén, razonadamente, las dimensiones que minimizan la superficie de papel.  \begin{solution}   \\ Función a optimizar y derivadas: \\ $\left[ f(x), \  \left(4 + \frac{18}{x}\right) \left(x + 2\right)\right]$ \\ $\left[ f'(x), \  4 - \frac{36}{x^{2}}\right]$ \\ $\left[ f''(x), \  \frac{72}{x^{3}}\right]$ \\Extremos relativos: \\ $\left[ MAXIMO, \  x, \  -3\right]$ \\ $\left[ MINIMO, \  x, \  3\right]$ \\\\ \\ \\  \\  \\   \end{solution} \part[1] 19. Halla las dimensiones de un depósito abierto superiormente en forma de prisma recto de base cuadrada, de 500 m 3 de
capacidad y que tenga un revestimiento de coste mínimo.  \begin{solution}   \\ Función a optimizar y derivadas: \\ $\left[ f(x), \  x^{2} + \frac{4 \cdot 500 x}{x^{2}}\right]$ \\ $\left[ f'(x), \  2 x - \frac{2000}{x^{2}}\right]$ \\ $\left[ f''(x), \  2 + \frac{4000}{x^{3}}\right]$ \\Extremos relativos: \\ $\left[ MINIMO, \  x, \  10\right]$ \\\\ \\ \\  \\  \\   \end{solution} \part[1] 20. Un triángulo isósceles, de perímetro 10 m, gira alrededor de la altura relativa al lado desigual engendrando un cono.
Halla la longitud de sus lados para que el cono tenga volumen máximo.  \begin{solution}   \\ Función a optimizar y derivadas: \\ $\left[ f(x), \  \frac{\pi \left(\frac{x}{2}\right)^{2} \sqrt{- \frac{x^{2}}{4} + \left(\frac{10 - x}{2}\right)^{2}}}{3}\right]$ \\ $\left[ f'(x), \  - \frac{5 \sqrt{5} \pi x \left(x - 4\right)}{24 \sqrt{5 - x}}\right]$ \\ $\left[ f''(x), \  \frac{5 \sqrt{5} \pi \left(3 x^{2} - 24 x + 40\right)}{48 \left(5 - x\right)^{\frac{3}{2}}}\right]$ \\Extremos relativos: \\ $\left[ MINIMO, \  x, \  0\right]$ \\ $\left[ MAXIMO, \  x, \  4\right]$ \\\\ \\ \\  \\  \\   \end{solution} \part[1] 21. Queremos vallar un campo rectangular que está junto a un camino. La valla del lado del camino cuesta 5 euros/m y la
de los otros tres lados, 0,625 euros/m. Halla el área del campo de mayor superficie que podemos cercar con 1800 euros.  \begin{solution}   \\ Función a optimizar y derivadas: \\ $\left[ f(x), \  \frac{x \left(1800 - 5.625 x\right)}{0.625 \cdot 2}\right]$ \\ $\left[ f'(x), \  1440.0 - 9.0 x\right]$ \\ $\left[ f''(x), \  -9.0\right]$ \\Extremos relativos: \\ $\left[ MAXIMO, \  x, \  160.0\right]$ \\\\ \\ \\  \\  \\   \end{solution} \part[1] 22. Los barriles que se utilizan para almacenar petróleo tienen forma cilíndrica y una capacidad de 160 litros. Halla las
dimensiones del cilindro para que la cantidad de chapa empleada en su construcción sea mínima.  \begin{solution}   \\ Función a optimizar y derivadas: \\ $\left[ f(x), \  2 \pi x^{2} + \frac{2 \cdot 160 \pi x}{\pi x^{2}}\right]$ \\ $\left[ f'(x), \  4 \pi x - \frac{320}{x^{2}}\right]$ \\ $\left[ f''(x), \  4 \pi + \frac{640}{x^{3}}\right]$ \\Extremos relativos: \\ $\left[ MINIMO, \  x, \  \frac{2 \sqrt[3]{10}}{\sqrt[3]{\pi}}\right]$ \\\\ \\ \\  \\  \\   \end{solution} \part[1] 33.- Entre todos los triángulos isósceles de perímetro 30 cm, ¿cuál es el de área máxima?
  \begin{solution}   \\ Función a optimizar y derivadas: \\ $\left[ f(x), \  \frac{x \sqrt{- \frac{x^{2}}{4} + \left(\frac{30 - x}{2}\right)^{2}}}{2}\right]$ \\ $\left[ f'(x), \  - \frac{3 \sqrt{15} \left(x - 10\right)}{4 \sqrt{15 - x}}\right]$ \\ $\left[ f''(x), \  \frac{3 \sqrt{15} \left(x - 20\right)}{8 \left(15 - x\right)^{\frac{3}{2}}}\right]$ \\Extremos relativos: \\ $\left[ MAXIMO, \  x, \  10\right]$ \\\\ \\ \\  \\  \\   \end{solution} \part[1] 34.- Se quiere construir un recipiente cónico de generatriz 10 cm y de capacidad máxima. ¿Cuál debe ser el radio de la base?
  \begin{solution}   \\ Función a optimizar y derivadas: \\ $\left[ f(x), \  \frac{\pi x^{2} \sqrt{- x^{2} + 10^{2}}}{3}\right]$ \\ $\left[ f'(x), \  \frac{\pi x \left(200 - 3 x^{2}\right)}{3 \sqrt{100 - x^{2}}}\right]$ \\ $\left[ f''(x), \  \frac{2 \pi \left(3 x^{4} - 450 x^{2} + 10000\right)}{3 \left(100 - x^{2}\right)^{\frac{3}{2}}}\right]$ \\Extremos relativos: \\ $\left[ MINIMO, \  x, \  0\right]$ \\ $\left[ MAXIMO, \  x, \  - \frac{10 \sqrt{6}}{3}\right]$ \\ $\left[ MAXIMO, \  x, \  \frac{10 \sqrt{6}}{3}\right]$ \\\\ \\ \\  \\  \\   \end{solution} \part[1] 35.- Con una lámina cuadrada de 10 dm de lado se quiere construir una caja sin tapa. Para ello, se recortan unos cuadrados
de los vértices. Calcula el lado del cuadrado recortado para que el volumen de la caja sea máximo. Si la altura de la caja no
puede pasar de 2 dm ¿cuál es la medida del lado del cuadrado que debemos recortar?
  \begin{solution}   \\ Función a optimizar y derivadas: \\ $\left[ f(x), \  x \left(10 - 2 x\right)^{2}\right]$ \\ $\left[ f'(x), \  4 \left(x - 5\right) \left(3 x - 5\right)\right]$ \\ $\left[ f''(x), \  24 x - 80\right]$ \\Extremos relativos: \\ $\left[ MAXIMO, \  x, \  \frac{5}{3}\right]$ \\ $\left[ MINIMO, \  x, \  5\right]$ \\\\ \\ \\  \\  \\   \end{solution} \part[1] 37.- Calcula la generatriz y el radio que debe tener un bote cilíndrico de leche condensada cuyo área total (incluyendo las
dos tapas) es de 150 cm 2 para que su volumen sea máximo.
  \begin{solution}   \\ Función a optimizar y derivadas: \\ $\left[ f(x), \  \frac{\pi x^{2} \left(150 - 2 \pi x^{2}\right)}{2 \pi x}\right]$ \\ $\left[ f'(x), \  - 3 \pi x^{2} + 75\right]$ \\ $\left[ f''(x), \  - 6 \pi x\right]$ \\Extremos relativos: \\ $\left[ MINIMO, \  x, \  - \frac{5}{\sqrt{\pi}}\right]$ \\ $\left[ MAXIMO, \  x, \  \frac{5}{\sqrt{\pi}}\right]$ \\\\ \\ \\  \\  \\   \end{solution} \part[1] 38.- Dos postes de 12 y 18 m de altura distan entre sí 30 m, Se desea tender un cable uniendo un punto del suelo entre los
dos postes con los extremos de éstos. ¿Dónde hay que situar el punto del suelo para que la longitud total del cable sea
mínima?
  \begin{solution}   \\ Función a optimizar y derivadas: \\ $\left[ f(x), \  \sqrt{x^{2} + 12^{2}} + \sqrt{\left(30 - x\right)^{2} + 18^{2}}\right]$ \\ $\left[ f'(x), \  \frac{x}{\sqrt{\left(x - 30\right)^{2} + 324}} + \frac{x}{\sqrt{x^{2} + 144}} - \frac{30}{\sqrt{\left(x - 30\right)^{2} + 324}}\right]$ \\ $\left[ f''(x), \  - \frac{x^{2}}{\left(x^{2} + 144\right)^{\frac{3}{2}}} - \frac{\left(x - 30\right)^{2}}{\left(\left(x - 30\right)^{2} + 324\right)^{\frac{3}{2}}} + \frac{1}{\sqrt{\left(x - 30\right)^{2} + 324}} + \frac{1}{\sqrt{x^{2} + 144}}\right]$ \\Extremos relativos: \\ $\left[ MINIMO, \  x, \  12\right]$ \\\\ \\ \\  \\  \\   \end{solution}
        \end{parts}
        \end{multicols}
        
    \end{questions}
    \end{document}
    