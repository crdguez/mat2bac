\documentclass[addpoints,spanish, 12pt,a4paper]{exam}
%\documentclass[answers, spanish, 12pt,a4paper]{exam}
%\printanswers
\pointpoints{punto}{puntos}
\hpword{Puntos:}
\vpword{Puntos:}
\htword{Total}
\vtword{Total}
\hsword{Resultado:}
\hqword{Ejercicio:}
\vqword{Ejercicio:}

\usepackage[utf8]{inputenc}
\usepackage[spanish]{babel}
\usepackage{eurosym}
%\usepackage[spanish,es-lcroman, es-tabla, es-noshorthands]{babel}


\usepackage[margin=1in]{geometry}
\usepackage{amsmath,amssymb}
\usepackage{multicol}
\usepackage{yhmath}

\pointsinrightmargin % Para poner las puntuaciones a la derecha. Se puede cambiar. Si se comenta, sale a la izquierda.
\extrawidth{-2.4cm} %Un poquito más de margen por si ponemos textos largos.
\marginpointname{ \emph{\points}}

\usepackage{graphicx}
\graphicspath{{../img/}} 

\newcommand{\class}{2º Bachillerato CIT}
\newcommand{\examdate}{\today}
\newcommand{\examnum}{Global Análisis}
\newcommand{\tipo}{A}


\newcommand{\timelimit}{50 minutos}

\renewcommand{\solutiontitle}{\noindent\textbf{Solución:}\enspace}


\pagestyle{head}
\firstpageheader{\includegraphics[width=0.2\columnwidth]{header_left}}{\textbf{Departamento de Matemáticas\linebreak \class}\linebreak \examnum}{\includegraphics[width=0.1\columnwidth]{header_right}}
\runningheader{\class}{\examnum}{Página \thepage\ of \numpages}
\runningheadrule


\begin{document}

\noindent
\begin{tabular*}{\textwidth}{l @{\extracolsep{\fill}} r @{\extracolsep{6pt}} }
\textbf{Nombre:} \makebox[3.5in]{\hrulefill} & \textbf{Fecha:}\makebox[1in]{\hrulefill} \\
 & \\
\textbf{Tiempo: \timelimit} & Tipo: \tipo 
\end{tabular*}
\rule[2ex]{\textwidth}{2pt}
Esta prueba tiene \numquestions\ ejercicios. La puntuación máxima es de \numpoints. 
La nota final de la prueba será la parte proporcional de la puntuación obtenida sobre la puntuación máxima. 

\begin{center}


\addpoints
 %\gradetable[h][questions]
	\pointtable[h][questions]
\end{center}

\noindent
\rule[2ex]{\textwidth}{2pt}

\begin{questions}

\question \textbf{(Sept. 03)} Dada la función $f(x)=x \ln x$, calcula:

\begin{parts}
\part[1] Su dominio
\begin{solution}
$(0,+\infty)$
\end{solution}
\part[1] Sus ceros
\begin{solution}
$x=0$ y $x=1$
\end{solution}
\part[1] Sus extremos relativos
\begin{solution}
Mínimo relativo en $(\frac{1}{e},-\frac{1}{e})$
\end{solution}
\end{parts}

\question \textbf{(Junio 99)} Dada la función $$f(x)=\begin{cases} 2x^{2} + a x + b & si \ x \leq 1 \\ \ln x -1 & si \ x > 1\end{cases}$$ 

\begin{parts}
\part[1] Encontrar los valores de a y b para que la función sea continua y su gráfica pase por el origen de 
coordenadas.
\begin{solution}

\end{solution}
\part[1] Estudiar su derivabilidad
\begin{solution}

\end{solution}
\part[1] Hallar los puntos de su gráfica en los que 
la tangente es paralela al eje OX 
\begin{solution}

\end{solution}
\end{parts}

\question[2] \textbf{(Junio 12)} Halla el valor de $k$ para que $$\lim_{x \to 0}\left(\frac{e^x-e^{-x}+kx}{x- sen{x}}\right)=2$$
\begin{solution}
$k=-2$
\end{solution}

%\question[2] \textbf{(EVAU Junio 98)} Un  jardinero  dispone  de  120  metros  de  valla y  desea  delimitar  un  terreno  rectangular  y dividirlo en cinco lotes con vallas paralelas a uno de los lados del rectángulo. ¿Qué dimensiones debe tener el terreno para que el área sea la mayor posible? 
%\begin{solution} $2x+2y+4y=120 \to f(x)=x(6-3x) \to f'(x)=-6x+60 \ to $ Largo: 30 m.,  ancho: 10 m. \end{solution}

\question[3] \textbf{(Junio 98)} Un campo de atletismo de 400 metros de perímetro consiste en un rectángulo con un semicírculo en cada uno de dos lados opuestos. Hallar las dimensiones del campo para que el área de la parte rectangular sea lo  mayor posible 
\begin{solution} $f(x)=x\cdot \frac{400 - 2x}{\pi} \to f'(x)=\frac{1}{\pi}(400-4x)\to $
100 mts y $\frac{200}{\pi}$ mts
\end{solution}

\question[3] \textbf{(Junio 12)} Calcula la siguiente integral indefinida:

$$\int \frac{x^2+11x}{x^3-2x^2-2x+12} dx$$

\question[2]  \textbf{(Junio 98)} Calcular el área del recinto limitado por la curva  $y = x e^x$, el eje OX, el eje OY y la recta paralela al eje OY que pasa por el punto donde la curva tiene su mínimo relativo.
\begin{solution}
Mínimo en $(-1 , -0.37)$

Área: $\int_{-1}^0 xe^xdx=1-\frac{2}{e}\approx 0.26 / ud^2$
\end{solution}

\addpoints



\end{questions}

\end{document}
\grid
